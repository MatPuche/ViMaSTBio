Ce projet de groupe a pour but de nous faire travailler à plusieurs sur des notions vues en cours afin de les approfondir et les appliquer à un cas concret. Il a  été réalisé en groupe de cinq composé de deux étudiants en SI et trois en GI, ce qui a permis une polyvalence des connaissances. 
\newline 

M. Samuel Buchet et M. Olivier Roux travaillent actuellement sur une analyse de données de régulations biologiques. Pour ce faire, ils ont utilisé jusqu'à présent des représentations faites sur papier, rendant le travail fastidieux et difficile à exploiter du fait du nombre considérable de données et de leur évolution rapide. C'est dans ce contexte qu'ils nous ont demandé de créer un outil interactif permettant la manipulation et la visualisation des données biologiques qu'ils étudient. 
