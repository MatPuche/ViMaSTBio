Les objectifs initialement fixés ont pour une grande majorité été atteints et notre outil est ainsi fonctionnel mais il reste plusieurs améliorations que nous aurions aimé apporter à celle-ci. 

D’abord, nous souhaitions faire une représentation de l'évolution des gènes supposée par les automates sous forme de chronogramme discrétisé. N’ayant pas eu le temps de nous attaquer à cette partie, nous pensons qu’il serait intéressant qu’un futur groupe reprenne le projet afin d’en implémenter cette partie. Cela imposerait de réaliser un programme de décision qui choisirait de franchir les transitions de manière aléatoire dans un premier temps. Un premier chronogramme “prédit” serait alors dessiné. Puis, on peut également imaginer que la possibilité de choisir quelle transition franchir pourrait être donnée à l’utilisateur afin qu’il puisse observer les évolutions qu’il souhaite. Dans le cadre d’une étude biologique, cette fonctionnalité apparaît être intéressante. 

Ensuite, nous avons appris en fin de projet que de nouveaux arcs pourraient être à représenter : ceux dont l’état initial et celui d’arrivée sont le même. Nous n’avons pu modifier le code de telle sorte qu’ils soient affichés joliment et pensons donc qu’il serait possible d’améliorer cette représentation. 

Deux points précis liés aux fenêtres peuvent être corrigés. D’une part, déplacer un élément sélectionne systématiquement une partie du texte présent sur la page à cause du comportement par défaut des navigateurs, mais désactiver ce comportement par défaut empêche d’interagir avec le contenu du graphe (zoom, déplacement) ou avec certains objets html présents dans la page. Nous n’avons pas trouvé comment résoudre ce problème qui ne provoque pas d’erreurs mais peut être légèrement gênant à terme. D’autre part, les fenêtres sont un peu plus dures à redimensionner lorsqu’elles contiennent des barres de défilement puisque la marge de sélection de la bordure est fortement réduite.
Il serait par ailleurs intéressant de rajouter la possibilité de déplacer et de redimensionner le graphe lui-même (comme on peut le faire avec les graphes des tableurs par exemple).
