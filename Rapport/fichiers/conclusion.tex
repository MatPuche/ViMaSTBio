Ce projet a été l'occasion pour nous de consolider nos acquis des cours d'informatique et surtout de toucher à plusieurs domaines inconnus jusque là de la programmation au travers d'un besoin réel. Une grande partie du projet a été consacrée à de la recherche, des tests et de la lecture de documentation afin de comprendre le fonctionnement des différents outils que nous avons manipulés. 

Ainsi, nous avons eu l'occasion d'apprendre comment interagir avec les API (majoritairement celles de Google) pour les logiciels les fournissant et de trouver une alternative pour les autres, via l'envoi de requêtes POST pour le site Doodle en l'occurrence. 

D'autre part, nous avons appris à concevoir une application Web dans son ensemble via le framework Flask. Pour le déploiement, nous avons suivi plusieurs pistes sans succès, notamment via Firebase, mais nous nous sommes finalement accordés sur le choix de Heroku, service le plus adapté aux applications Flask. 

Enfin, les différents problèmes rencontrés au cours du projet nous ont permis d'apprendre à trouver des alternatives notamment lorsque la solution n'était pas immédiate. Par exemple, le problème de l'authentification d'un compte Google nous a valu plusieurs heures de recherches et de tests avant de trouver une solution. 

Ce projet est donc un projet polyvalent qui nous a permis de developper des connaissances dans de nombreux domaines. 