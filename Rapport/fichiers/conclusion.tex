Nous avons ainsi réalisé une première version de l’outil permettant aux biologistes d'obtenir une représentation plus claire de leurs données. Ce projet a été très formateur sur plusieurs aspects. 

D’abord, il a permis à chacun de nous d’approfondir ses connaissances en conception de site et notamment en représentation graphiques et dessin. N’ayant jamais codé en JavaScript auparavant, nous nous sommes formés à ce langage et avons découvert le grand nombre de possibilités qu’il permet. De plus, ce projet nous a donné l’occasion de manipuler la librairie p5 et d’en appréhender les fonctionnalités. 

Aussi, c’était pour nous l’une des premières fois qu’un projet à cinq nous était confié, ce qui s’est avéré être un vrai challenge de travail d’équipe. En effet, la répartition des  tâches et la mise en commun n’étaient pas évidentes au début mais nous sommes petit à petit parvenus à séparer le projet en deux et à donner à chacun une part de travail.En somme, nous avons surtout appris à partager du code au sein d’un même projet.

Enfin, une grande partie du projet a été consacrée à de la recherche, des tests et de la lecture de documentation afin de comprendre le fonctionnement des différents outils que nous avons manipulés. Nous nous sommes ainsi confronté à la réalité de la programmation qui nécessite bien souvent de nombreuses recherches avant d’arriver à un résultat fonctionnel.
